\documentclass[12pt]{article}
\usepackage{ucs}
\usepackage[utf8x]{inputenc}
\usepackage[russian]{babel}
\begin{document}
Фольклор~--- явление очень интересное и многоплановое, не
перестающее занимать умы исследователей. Каких только
способов не изобретают люди, чтобы повеселиться. Вот,
например, неизвестяый автор взял два серьезных стихотворения
двух поэтов-классиков и сделал из них своеобразный винегрет:

\bigskip
\bigskip
\noindent Однажды, в студеную зимнюю пору, \\
Сижу за решеткой в темнице сырой.\\
Тияху, поднимается медленно в гору\\
Вскормленный в неволе орел молодой,\\
И, шествуя важно, в спокойствии чинном\\
Мой грустный товарищ, махая крылом,\\
В больших сапогах, в полушубке овчинном\\
Кровавую пицу клюет под окном.
\bigskip

Вот такое вот "<народное творчество>", извольте видеть.
\end{document}